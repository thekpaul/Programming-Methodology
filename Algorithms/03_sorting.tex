\chapter{Various Ways of Sorting}
The objective of this callenge is sorting given algebraic (in this case,
\texttt{double}) elements of an aray in incrasing order. Some elements may be of
equal value.

\section{Quick Sort}
\inputminted[firstline=5,lastline=17,linenos,xleftmargin=0.25in]{cpp}
    {../Algorithms/sorting.cpp}
\inputminted[firstline=87,lastline=89,linenos,xleftmargin=0.25in]{cpp}
    {../Algorithms/sorting.cpp}
\inputminted[firstline=91,lastline=94,linenos,xleftmargin=0.25in]{cpp}
    {../Algorithms/sorting.cpp}
\inputminted[firstline=120,lastline=121,linenos,xleftmargin=0.25in]{cpp}
    {../Algorithms/sorting.cpp}
Quick Sort is a recursive algorithm that does the following:
\begin{enumerate}
\item Choose a ``pivot" element: \texttt{Line 6}
\item Swap elements that are larger than the pivot with elements that are
    smaller but on the righthand side of the chosen element: \texttt{Lines 8:14}
\item Continue until all elements are sorted, then recursively proceed with the
    left and right subarrays of the pivot: \texttt{Lines 15, 16}
\end{enumerate}
Using two half-recursions results in the following time complexity analysis: \[
T(n)=2T\left(\frac{n}{2}\right)+n=2^kT\left(\frac{n}{2^k}\right)+kn=O(n\log n)\]

\section{Merge Sort}
\inputminted[firstline=19,lastline=37,linenos,xleftmargin=0.25in]{cpp}
    {../Algorithms/sorting.cpp}
\inputminted[firstline=87,lastline=89,linenos,xleftmargin=0.25in]{cpp}
    {../Algorithms/sorting.cpp}
\inputminted[firstline=96,lastline=99,linenos,xleftmargin=0.25in]{cpp}
    {../Algorithms/sorting.cpp}
\inputminted[firstline=120,lastline=121,linenos,xleftmargin=0.25in]{cpp}
    {../Algorithms/sorting.cpp}
Merge Sort is also a recursive algorithm that depends on two half-recursions, so
the time complexity analysis goes: \[ T(n)=2T\left(\frac{n}{2}\right)+n=
2^kT\left(\frac{n}{2^k}\right)+kn=O(n\log n) \]

\section{Insertion Sort}
\inputminted[firstline=46,lastline=55,linenos,xleftmargin=0.25in]{cpp}
    {../Algorithms/sorting.cpp}
\inputminted[firstline=87,lastline=89,linenos,xleftmargin=0.25in]{cpp}
    {../Algorithms/sorting.cpp}
\inputminted[firstline=101,lastline=104,linenos,xleftmargin=0.25in]{cpp}
    {../Algorithms/sorting.cpp}
\inputminted[firstline=120,lastline=121,linenos,xleftmargin=0.25in]{cpp}
    {../Algorithms/sorting.cpp}
Insertion Sort is a double-loop algorithm, therefore time complexity analysis is
as follows: \[ T(n)=\sum_{i=1}^n\sum_{j=1}^i\epsilon=O(n^2) \]

\section{Stooge Sort}
\inputminted[firstline=57,lastline=67,linenos,xleftmargin=0.25in]{cpp}
    {../Algorithms/sorting.cpp}
\inputminted[firstline=87,lastline=89,linenos,xleftmargin=0.25in]{cpp}
    {../Algorithms/sorting.cpp}
\inputminted[firstline=106,lastline=109,linenos,xleftmargin=0.25in]{cpp}
    {../Algorithms/sorting.cpp}
\inputminted[firstline=120,lastline=121,linenos,xleftmargin=0.25in]{cpp}
    {../Algorithms/sorting.cpp}
Stooge Sort is a recursive algorithm of three subarrays. The time complexity is:
\[ T(n)=3T\left(\frac{3}{2}n\right)+1,~\therefore T(n)\approx
O\left(n^{2.7}\right) \]

\section{Heap Sort}
\inputminted[firstline=69,lastline=79,linenos,xleftmargin=0.25in]{cpp}
    {../Algorithms/sorting.cpp}
\inputminted[firstline=87,lastline=89,linenos,xleftmargin=0.25in]{cpp}
    {../Algorithms/sorting.cpp}
\inputminted[firstline=111,lastline=118,linenos,xleftmargin=0.25in]{cpp}
    {../Algorithms/sorting.cpp}
\inputminted[firstline=120,lastline=121,linenos,xleftmargin=0.25in]{cpp}
    {../Algorithms/sorting.cpp}
Heap sort is a loop-and-recursive algorithm of time complexy analysis \[ T(n)=
O(n\log n) \]
