\chapter{Matrix-Chain Multiplication Algorithm}
The objective of this challenge is to calculate the minimum amount of scalar
multiplications a group of matrices needs to return the complete and correct
matrix product. Consider three matrices $\mathbf{A}[10][100]$, $\mathbf{B}[100]
[5]$, and $\mathbf{C}[5][50]$. There are two ways to multiply these matrices:
\begin{enumerate}
\item $(\mathbf{A}\times\mathbf{B})\times\mathbf{C}$: $\mathbf{A}\times
    \mathbf{B}$ requires $10\times100\times5=5000$ scalar multiplications.
    $\mathbf{AB}\times\mathbf{C}$ requires $10\times5\times50=2500$. Therefore
    the total amount of scalar multiplications is $5000+2500=7500$.
\item $\mathbf{A}\times(\mathbf{B}\times\mathbf{C})$: $\mathbf{B}\times
    \mathbf{C}$ requires $100\times5\times50=25000$ scalar multiplications.
    $\mathbf{A}\times\mathbf{BC}$ requires $10\times100\times50=50000$.
    Therefore the total amount of scalar multiplications is $25000+50000=75000$.
\end{enumerate}

Given a chain of $n$ matrices and an integer array of $n+1$ elements where each
matrix $\mathbf{A_i}$ shows the dimesion of $\left(p_{i-1},~p_i\right)$, the
following solutions are algorithms that minimise scalar multiplication amounts.

\section{Optimal Solution}
\imb[4][26]{\alg/matrixchain.cpp}

Matrix chain multiplication has $T(n)=\Theta\left(n^2\right)$ subproblems, each
featuring no more than $n-1$ choices. Therefore the time complexity of this
algorithm is: \[ T(n)=\Theta\left(n^3\right) \]
