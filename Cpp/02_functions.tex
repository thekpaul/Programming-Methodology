\chapter{Functions}

\section{Function Definition}
A function definition consists of four major parts:
\begin{enumerate}
\item Function Type: The type of value that the function returns. If a function
    returns an integer value (as shown in Figure 2.1), then the function type is
    \imc{int}. Function that do not return a value (as shown is Figure 2.2) are
    type-defined as \imc{void}.
\item Function Name: The name used to call the process that is the function.
\item Function Parameters: The arguments a function requires in order to return
    corresponding values. Parameters are declared with data types and variable
    names that are used to indicate said parameters within the function.
\item Function Process: The executed process when the function is called. If the
    function is not a \imc{void} function, the process ends by returning a value
    that is correspondent to the type of the function.
\end{enumerate}
\begin{minipage}{.5\textwidth}
\begingroup
\imb[3][7]{\cpp/02_functions.cpp}[An \imc{int} function]
\endgroup
\end{minipage}
\begin{minipage}{.5\textwidth}
\begingroup
\imb[8][12]{\cpp/02_functions.cpp}[A \imc{void} function]
\endgroup
\end{minipage}

\section{Argument Passing Mechanism}
\begin{enumerate}
\item Procedure Abstraction: A function encapsulates and names an algorithm to
    solve a particular problem.
\item Default Arguments: Default values can be given for each function parameter
    by asssigning the variables beforehand. If no new values are detected when
    concerned function is called, the default values are automatically used.
    \begingroup
    \imb[14][18]{\cpp/02_functions.cpp}[Function with Default Arguments]
    \endgroup
\item Passing Arguments: Arguments can either be called by value, reference, or
    pointer:
    \begin{enumerate}
    \item Call by \textbf{Value}: Function copies the value of provided variable
        but doesn't use the variable itself. Therefore changes made within the
        function, when called by value, \textbf{does not affect the variable}.
        C++, by default, passes arguments by value. No additional symbols need
        be preceded.
    \item Call by \textbf{Reference}: Function uses the variable itself, and
        therefore changes made within the function \textbf{affects the variable}
        during and after processing the function. C++ requires variables to be
        prepended by an \textbf{ampersand} (\imc{&}) to call by reference.
    \item Call by \textbf{Pointer}: Function uses the variable itself as well,
        but unlike references, which are essentially memory addresses of values,
        pointers are variables that hold these memory addresses and can be
        re-assigned to a different value at any point during the program. C++
        requires variables to be prepended by an \textbf{asterick} (\imc{*}) to
        call by pointer.
    \end{enumerate}
    \begin{minipage}{.5\textwidth}
    \imb[19][31]{\cpp/02_functions.cpp}[Functions calling variables by value,
    reference, and pointer] \end{minipage}
    \begin{minipage}{.5\textwidth}
    \imb[54][63]{\cpp/02_functions.cpp}[Calling each function in the \imc{main}
    function] \end{minipage}
\end{enumerate}

\section{Inline Functions}
Inline functions are functions that are essentially copied and pasted into lines
when they are called. \begingroup
\imb[33][36]{\cpp/02_functions.cpp}[Example of an Inline Function]
\endgroup

\section{Recursive Functions}
Recursive Function: A function that calls itself, either directly or indirectly,
during the process of its own execution. Recursion is used when loops that use
different, incremental or decremental arguments are required. \begingroup
\imb[38][43]{\cpp/02_functions.cpp}[Example of a Recursive Function] \endgroup
