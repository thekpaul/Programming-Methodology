\chapter{Functions}

\section{Function Definition}
A function definition consists of four major parts:
\begin{enumerate}
\item Function Type: The type of value that the function returns. If a function
    returns an integer value (as shown in Figure 2.1), then the function type is
    \imc{int}. Function that do not return a value (as shown is Figure 2.2) are
    type-defined as \imc{void}.
\item Function Name: The name used to call the process that is the function.
\item Function Parameters: The arguments a function requires in order to return
    corresponding values. Parameters are declared with data types and variable
    names that are used to indicate said parameters within the function.
\item Function Process: The executed process when the function is called. If the
    function is not a \imc{void} function, the process ends by returning a value
    that is correspondent to the type of the function.
\end{enumerate}

% \imb[<++>][<++>]{\cpp/02_functions.cpp}
% \imb[<++>][<++>]{\cpp/02_functions.cpp}

\section{Procedural Abstraction}

\section{Argument Passing Mechanism}

\section{Inline Functions}

\section{Recursive Functions}
