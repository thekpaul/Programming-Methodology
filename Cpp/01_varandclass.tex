\chapter{Variables and Class}

\section{The Basics}
\begingroup
\imb[1][4]{\cpp/01_varandclass.cpp}
\imb[29][30]{\cpp/01_varandclass.cpp}[Printing \texttt{Hello, World!} with C++]
\endgroup
C++ is the younger, more powerful and versatile form of what C was meant to be.
C++ utilises features such as object-orientated programming not supported by C.

\subsection{Commenting}
Single-line comments use two forward slashes: \imc{// Comment 1}\\
Multi-line comments use forward slash and star at each end to denote beginning
and end: \imc{/* Comment 2 */}

\subsection{Types and Variables}
\begin{enumerate}
\item Primitive, built-in types:
    \begin{enumerate}
    \item \imc{void} is used to determine functions and variables of no return
        value.
    \item \imc{bool, char, int, float, double} and et cetra are used to return
        certain values.
    \item \imc{unsigned} is prepended to targets that are always of positive
        value.
    \end{enumerate}
\item Enumerations: \imc{enum} is used to define groups of integer constants.
    \begingroup
    \imb[5]{\cpp/01_varandclass.cpp}[An example declaration of \imc{enum}]
    \endgroup
\item Variable Scopes: Varaibles are defined only up to the level of their own
    declaration.
    \begin{enumerate}
    \item Global Variables: Variables that are declared outside functions and
        classes. Global variables can be defined as \imc{static} to make its
        scope local to one file.
    \item Local Variables: Variables that are declared within a function, class,
        or otherwise enclosed snippets of code. Variables can only be used and
        memoized within that specific snippet. However, these variables can also
        be defined as \imc{static} to retain its value from previous uses and
        bypass re-initialisation.
    \end{enumerate}
\item Constant Qualifier: Variables preceded by the keyword \imc{const} cannot
    be altered by the program. \begingroup
    \imb[6]{\cpp/01_varandclass.cpp}[An example declaration of constant integer]
    \endgroup
\end{enumerate}

\section{Basic Expressions and Statements}
\begin{enumerate}
\item Basic Expressions: Expressions are combinations of iterated variables
    and operators that portray a new value or meaning.
    \begin{enumerate}
    \item Arithmetic Expressions: Like most other programming languages, C++
        features arithmetic operators \verb|+, -, *, /, %| which each perform
        additions, subtractions, multiplications, truncating divisions and
        modular calculations. \begingroup
        \imb[7][8]{\cpp/01_varandclass.cpp}[Arithmetic operators in C++]
        \endgroup
    \item Relational Expressions: Like most other programming languages, C++
        features relational operators \imc{==, !=, <, >, <=, >=} which each mean
        equality, inequality, less-than, greater-than, less-than-or-equal and
        greater-than-or-equal. \begingroup
        \imb[9][10]{\cpp/01_varandclass.cpp}[Relational operators in C++]
        \endgroup
    \end{enumerate}
\item Basic Statements: Statements are combinations of expressions and snippets
    of code that portray the actions to be taken by the program itself.
    \begin{enumerate}
    \item Conditional Statements: Like most other programming languages, C++
        features selective execution of code with keywords \imc{if...else} and
        \imc{switch}. \imc{if...else} are used for extensive description of the
        determining criteria, while \imc{switch} is used for multiple possible
        outcomes of the same, relatively simple value. \begingroup
        \imb[11][17]{\cpp/01_varandclass.cpp}[Conditional Statements in C++]
        \endgroup
    \item Ternary Conditional Operator: The only operator of the C++ language to
        feature three operands, conditional operator \imc{expr1 ? expr2 : expr3}
        evaluates the boolean value of \imc{expr1}, and if it is true executes
        \imc{expr2}, else \imc{expr3}. \begingroup
        \imb[18][19]{\cpp/01_varandclass.cpp}[The Ternary Operator of C++]
        \endgroup
    \item Iterative Statements: Like other programming languages, C++ features
        iterative execution of code with keywords \imc{for} and \imc{while}.
        \begingroup
        \imb[20][26]{\cpp/01_varandclass.cpp}[Iterative Statements in C++]
        \endgroup
    \end{enumerate}
\item Memory Management: Keywords \imc{new} and \imc{delete} each process the
    allocation and deallocation of memory in C++. \begingroup
    \imb[27][28]{\cpp/01_varandclass.cpp}[Allocation of Memory in C++]
    \endgroup
\end{enumerate}
